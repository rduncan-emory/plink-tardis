% partition-across-genome.tex

%-------------------------------------------------------------
\startsubsubject[title={Partitioning SNP indicies across the genome}]
%-------------------------------------------------------------

Another Perl script,
\filename{partition-snps-genome.pl},
has been developed generate \PLINK commands for a partitioning across the genome
with a specified number of SNPs to each batch.
This is accomplished internally by calling the previous script
\filename{partition-snps-by-chromosome.pl} for each chromosome
in the genome calculating the number of partitions required for each.
Eventually, the commands generated by this set of scripts
will be executed separately by the scheduler.
%
Using the \command{--help} option produces a more detailed description of this
script and the options passed to it.
%%
\startSTDIN
./partition-snps-genome.pl --help
\stopSTDIN
%%

To print a count of SNPs across every for a given set of inputs, use the \command{--summary} option.
In this example, a summary of SNP counts for each chromosome will be printed for the case where
\command{--M=100} SNPs are to be allocated to each partition.
%%
\startSTDIN
./partition-snps-genome.pl --options plink-options --M=100 --summary
\stopSTDIN
%%
The output for this command is:
%%
\startSTDOUT
chromosome 01 with 53395 SNPs:   100 per partition 
chromosome 02 with 52558 SNPs:   100 per partition 
chromosome 03 with 43361 SNPs:   100 per partition 
chromosome 04 with 36428 SNPs:   100 per partition 
chromosome 05 with 37879 SNPs:   100 per partition 
...
chromosome 22 with  9484 SNPs:   100 per partition 
\stopSTDOUT
%%

Finally, to produce the actual \PLINK commands for this partitioning scheme,
redirecting the output to the file \filename{cmds}:
%%
\startSTDIN
./partition-snps-genome.pl --options plink-options --M=100 > cmds
\stopSTDIN
%%
The contents of \filename{cmds} is similar to the example provided in the previous section,
except that \PLINK commands are produced for every chromosome:
%%
\startSTDOUT
plink --noweb  --bfile=AA --covar=cov.dat --R=Rplink.R --silent --chr 1 --snps rs3094315-rs12142199 --out=BDI_chr01_rs1
plink --noweb  --bfile=AA --covar=cov.dat --R=Rplink.R --silent --chr 1 --snps rs10449893-rs884080 --out=BDI_chr01_rs2
plink --noweb  --bfile=AA --covar=cov.dat --R=Rplink.R --silent --chr 1 --snps rs7513222-rs2494641 --out=BDI_chr01_rs3
plink --noweb  --bfile=AA --covar=cov.dat --R=Rplink.R --silent --chr 1 --snps rs4648843-rs6683516 --out=BDI_chr01_rs4
...
plink --noweb  --bfile=AA --covar=cov.dat --R=Rplink.R --silent --chr 22 --snps rs1053744-rs3888396 --out=BDI_chr22_rs94
\stopSTDOUT
%%

Another script, described next, will be used to produce the corresponding set of batch scripts while
maintaining a master script containing a queueing command for each script.

\stopsubsubject

\endinput
