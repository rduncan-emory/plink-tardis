% partition-by-chromosome.tex

%-------------------------------------------------------------
\startsubsubject[title={Partitioning SNP indicies along a chromosome}]
%-------------------------------------------------------------

A Perl script,
\filename{partition-snps-by-chromosome.pl},
has been developed to contiguously partition the SNP index set
subsequently creating the corresponding \PLINK command for each partition.
Eventually, these commands will be executed separately by the scheduler.
%
Using the \command{--help} option produces a more detailed description of this
script and the options passed to it.
%%
\startSTDIN
./partition-snps-by-chromosome.pl --help
\stopSTDIN
%%

To print a count of SNPs across a given chromosome, use the \command{--summary} option.
In this example, we ask for a summary of 
\command{--N=100} partitions of the 
chromosome 2 SNPs listed in file \filename{AA.bim}
specified by the \command{bfile} line in the file \filename{plink-options}:

%%
\startSTDIN
./partition-snps-by-chromosome.pl --options plink-options --N 100 --chr 2 --summary
\stopSTDIN
%%
\startSTDOUT
SNP count in chromosome 2:   52558
SNP count in each partition:  525
\stopSTDOUT
%%
Thus, there are 52,558 SNPs along chromosome 2 in this data set and
partitioning into 100 subsets yields 525 SNPs in each batch.
The command to construct then store the set of corresponding \PLINK command calls into
a file named \filename{cmds} is:
%%
\startSTDIN
./partition-snps-by-chromosome.pl --options plink-options --N 100 --chr 2 > cmds
\stopSTDIN
%%
The output of this script is printed to the file \filename{cmds} consisting of the list of \PLINK commands for each partition preceded by a brief summary of SNP counts.  Within each of those commands the \command{--snps} option passes to \PLINK a beginning and ending SNP for each respective \PLINK call.  The output file looks like this:
%%
\startSTDOUT
plink --noweb  --bfile=AA --covar=cov.dat --R=Rplink.R --silent --chr 2 --snps rs10195681-rs11693178 --out=BDI_chr02_rs1
plink --noweb  --bfile=AA --covar=cov.dat --R=Rplink.R --silent --chr 2 --snps rs985467-rs1129241 --out=BDI_chr02_rs2
plink --noweb  --bfile=AA --covar=cov.dat --R=Rplink.R --silent --chr 2 --snps rs10197283-rs16863417 --out=BDI_chr02_rs3
plink --noweb  --bfile=AA --covar=cov.dat --R=Rplink.R --silent --chr 2 --snps rs7577322-rs10172433 --out=BDI_chr02_rs4
\stopSTDOUT
%%$
The \command{--noweb} option is automatically included in the options list.

Furthermore, we could run this for any other chromosomes of interest, appending to the file the resulting set of commands:
%%
\startSTDIN
./partition-snps-by-chromosome.pl --options plink-options --N 100 --chr 3 >> cmds
\stopSTDIN
%%
%%
\startSTDIN
./partition-snps-by-chromosome.pl --options plink-options --N 100 --chr 4 >> cmds
\stopSTDIN
%%
etc.

Note the difference in the use of '$>$~\filename{cmds}' in the first call to the script and '$>>$~\filename{cmds}' in subsequent calls.  
The former overwrites any old file contents or creates a new file with the output.  
The latter appends the outputs to the already existing file.  Thus, the commands for all chromosomes can be maintained in a single file.

Consequently, with potentially thousands of commands to queue, 
e.g., $\sim$2200 such commands to cover all SNPs across every chromosome, 
this would be difficult to manage manually.
Furthermore, with differing SNP counts for each chromosome, manually choosing the number of partitions
in this way is not an optimal strategy for working with the entire genome.
Next we will discuss a script for partitioning genome-wide sets of SNPs using a sensible uniform-count partitioning scheme.



\stopsubsubject

\endinput
