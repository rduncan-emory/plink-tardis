% update-software-environment.tex

%-------------------------------------------------------------
\startsubsubject[title={Updating the software environment}]
%-------------------------------------------------------------

First check to see if the most recent version of R is loaded into your software environment:
%%
\startSTDIN
  which R && R --version
\stopSTDIN
%%
Here is sample output from this command:
%%
\startSTDOUT
[rduncan-emory@tardis-6 ~]$ which R && R --version
/usr/local/packages/R/2.13.1/gcc-4.4.5/bin/R
R version 2.13.1 (2011-07-08)
Copyright (C) 2011 The R Foundation for Statistical Computing
ISBN 3-900051-07-0
Platform: x86_64-unknown-linux-gnu (64-bit)

R is free software and comes with ABSOLUTELY NO WARRANTY.
You are welcome to redistribute it under the terms of the
GNU General Public License version 2.
For more information about these matters see
http://www.gnu.org/licenses/.
\stopSTDOUT
%%$

The version of R loaded into the software environment here (2.13.1) is rather old.  
We can remove that version from the software stack
then add the latest version using the \command{module} command:
%%
\startSTDIN
  module rm R
  module add R/3.0.1
\stopSTDIN
%%

Similarly, check to see if \PLINK is available:
%%
\startSTDIN
  which plink
\stopSTDIN

%%
\startSTDOUT
[rduncan-emory@tardis-6 ~]$ which plink
/usr/bin/which: no plink in (/usr/local/packages/R/3.0.1//bin/:/usr/local/packages/rstudio/0.97.551/R-3.0.1/bin:...)
\stopSTDOUT %$
%%
If not, then add it to the software environment:
%%
\startSTDIN
  module add plink
\stopSTDIN
%%

\PLINK makes calls to R through the Rserve daemon which will be considered next.

\stopsubsubject

\endinput
