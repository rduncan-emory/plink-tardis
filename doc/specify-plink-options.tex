% specify-plink-options.tex

%-------------------------------------------------------------
\startsubsubject[title={Specifying common \PLINK command line options}]
%-------------------------------------------------------------

The command line options used by \PLINK should be stored in a two column file where 
(i) the first column is the option name and 
(ii) the second column is the corresponding value set by that option if appropriate.
Some options will not have a second column entry, e.g., the \command{--silent} option.
Any line beginning with the '\#' character will be ignored.

Here is the example options file, named \filename{plink-options}
that will be used in the examples below:
%%
\startSTDOUT
#-------------------------
# file:  plink-options
#-------------------------
bfile    AA
covar    cov.dat
R        Rplink.R
out      BDI
silent
\stopSTDOUT
%%$

\stopsubsubject

\endinput
