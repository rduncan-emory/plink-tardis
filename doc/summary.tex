% summary.tex


%-------------------------------------------------------------
\startsubsubject[title={Putting it all together}]
%-------------------------------------------------------------

Thus the sequence of commands to perform a \PLINK analysis of the SNPS across all chromosomes
with \command{M=100} SNPs per batch 
may look like this:
%%
\startSTDIN
./partition-snps-genome.pl --options plink-options --M=100 > cmds
./make-plink-schedule.pl --in=cmds --out=plink-qsubs --template=run-plink.template
./plink-qsubs
\stopSTDIN
%%


%\warning{NB:  for now, this is untested!}
\warning{NB:  plink-qsubs still untested}

%-------------------------------------------------------------
\startsubsubject[title={To do}]
%-------------------------------------------------------------


\stopsubsubject

\endinput



%%
\placefigure[here][fig:partition-flow]
{   Sequence for generating \PLINK scripts
  and a corresponding executable schedule script for 
  SNPs partioned across the genome.
}
{
  \externalfigure[plink-cmd-partition.pdf][width=0.8\textwidth]
}
%%
